\section{Anexos}
\subsection{Glosario}
Base de datos : Conjunto de datos pertenecientes a un mismo contexto y almacenados sistemáticamente para su posterior uso.

Código fuente : Conjunto de líneas de texto con los pasos que debe seguir la computadora para ejecutar un programa.

Diagrama de clases : Es un tipo de diagrama de estructura estática que describe la estructura de un sistema mostrando las clases del sistema, sus atributos, operaciones (o métodos), y las relaciones entre los objetos.

ID : Identificador de seguridad.

Iteración : Repetición, reiteración.

Procesador : Unidad central de procesamiento (CPU) interpreta las instrucciones y procesa los datos de los programas de computadora.

Prototipo : Primer ejemplar que se fabrica de una figura, un invento u otra cosa, y que sirve de modelo para fabricar otras iguales, o molde original con el que se fabrica.

Sistema operativo : Conjunto de órdenes y programas que controlan los procesos básicos de una computadora y permiten el funcionamiento de otros programas.

Software : Conjunto de programas y rutinas que permiten a la computadora realizar determinadas tareas.