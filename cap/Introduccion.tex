\section{Introducción}
\subsection{Propósito}
Este documento se muestra el análisis, modelo de trabajo, tecnologías y diseño utilizado para la producción de la aplicación “Proyecto Juego Espacios Turísticos en 360°” desarrollada para la compañía MauleTec.

 El enfoque de este documento se encuentra en el área de programación de el proyecto desarrollado, por lo que se espera de los lectores tener un interés en la creación de aplicaciones móviles para un grupo de usuarios casuales. Si bien la información encontrada requiere conocimiento en terminología relacionada a ingeniería de software, es recomendada para lectores que cuenten con un mínimo de programación básica en el lenguaje de Kotlin, incluyendo conocimiento básico en el área de base de datos ya que los desarrolladores han determinado que el nivel de entrada de este documento es bajo. 
 
 La estructura de este informe comienza con la introducción del proyecto describiendo de forma concisa. Luego se presenta el equipo de trabajo y su metodología de producción. El tercer punto es un análisis técnico de la aplicación pasando por su funcionalidad y utilizando una amplia cantidad de material visual para representar el trabajo realizado. Cuarto es la presentación del prototipo para validar su funcionalidad y demostrar su uso de forma práctica. El quinto punto es el diseño trabajado durante el desarrollo, mostrando la evolución del proyecto y cómo se llegaron a ciertas decisiones. En el punto seis refiere a la implantación usando el código de fuente para ser representado. El informe finaliza con un glosario con terminología relacionada al área presentada.
\subsection{Descripción breve del problema}
El proyecto fue presentado por la compañía MauleTec, que entregó un documento base explicando la problemática y los requerimientos de la aplicación. En base a la información entregada en el documento base se concluyó la siguiente problemática. El equipo de mauleTec cuenta con la habilidad para generar imágenes en 360 grados, por lo que presenta la oportunidad de crear una aplicación utilizando esta tecnología. Al buscar un enfoque lúdico se llega a la conclusión de desarrollar una aplicación estilo juego de mesa, donde una cantidad finita de jugadores exploran un tablero virtual en el que pueden acceder a estas imágenes en 360 grados. Luego de que los usuarios vean estas imágenes se les entregará un grupo de palabras al azar y tendrán que competir para crear la mejor historia utilizando el paisaje presentado en la imagen y las palabras. Luego pasarán a una fase donde los mismos usuarios votarán por la mejor frase y el jugador con la mejor valoración recibirá una victoria, para luego pasar a otra iteración de la misma forma, el juego finaliza cuando uno de los usuarios llega al máximo de victorias. Del análisis se destacan 3 puntos que el grupo de desarrollo concluyó como puntos críticos.


El primero es la realización de una base de datos, que cuenta como la parte central para la creación de esa aplicación, funcionando como conexión entre los usuarios. Para esta área se contará con el conocimiento del equipo de programación para llegar a una conclusión de cómo implementarla, ya sea con el uso de aplicaciones externas o no.


El segundo siendo el diseño de la aplicación ya que solo se entregó una simple descripción de actividades básicas que requiere el software, lo que, aunque entrega una libertad al equipo desarrollador también le pide más trabajo en los aspectos más detallados de este. Para solucionar esta situación se le dará un enfoque en la preproducción del proyecto solo para llegar a una idea más desarrollada del producto final.


Finalmente, el tercero es el material audiovisual necesario para la creación del software con la necesidad de usar imágenes en 360. Tomando en cuenta que el equipo de desarrollo se encuentra en ciudades distintas y la situación mundial se tendrá que recurrir a la búsqueda de este material por internet, asegurándose de que este se encuentre disponible para su uso público.
