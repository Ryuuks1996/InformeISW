\section{Introducción}
\subsection{Propósito}
Este documento se muestra el modelo de trabajo utilizado para la creación de una aplicación con fines de entretener a su usuario fomentando sus habilidades creativas. Si bien la información encontrada requiere un mínimo conocimiento de programación básica y de base de datos, su nivel de entrada es bajo. Tomando en cuenta su propósito se recomienda a sus lectores tener un interés en lo que refiere a la creación de aplicaciones móviles para un grupo de usuarios casuales. Como lo muestra su índice, la estructura de este informe se basará en las tres áreas principales del desarrollo de aplicaciones, estas siendo programación, diseño y material audiovisual.
\subsection{Descripción breve del problema}
En base a la información entregada en el documento base para el desarrollo de la aplicación se encontraron 3 factores principales para una realización correcta del proyecto. 

El primero es la realización de una base de datos, que cuenta como la parte central para la creación de esa aplicación. Para esta área se contará con el conocimiento del equipo de programación para llegar a una conclusión de como implementarla, ya sea con el uso de aplicaciones externas o no.

 El segundo siendo el diseño de la aplicación ya que solo se entregó una simple descripción de actividades básicas que requiere el software, lo que, aunque entrega una libertad al equipo desarrollador también le pide mas trabajo en los aspectos más detallados de este. Para solucionar esta situación se le dará un enfoque en la preproducción del proyecto solo para llegar a una idea mas desarrollada del producto final.
 
Finalmente, el tercero es el material audiovisual necesario para la creación del software con la necesidad de usar imágenes en 360. Tomando en cuenta que el equipo de desarrollo se encuentra en ciudades distintas y la situación mundial se tendrá que recurrir a la búsqueda de este material por internet, asegurándose de que esta está disponible para su uso público.