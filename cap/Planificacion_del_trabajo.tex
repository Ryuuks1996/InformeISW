\section{Planificación del Trabajo}

\subsection{Descripción del grupo de trabajo}
A continuación se especificará el grupo de trabajo, la cual estará encargada del desarollo de la aplicación de conquista de espacios turisticos en 360°. Se especificará su ID, nombre, conocimientos, rol y contacto de cada uno de los integrantes del grupo de trabajo.

\begin{table}[H]
    \centering
        \begin{tabular}{|l | p{12cm} |}        
        \hline
        \textbf{ID} & FD \\
        \hline
        \textbf{Nombre} & Felipe Durán \\
        \hline
        \textbf{Conocimientos} & Experiencia en lenguaje de programación como Phython, C, C++, C\# , Java, JavaScript, Kotlin y Conocimientos con base de datos MySQL. \\
        \hline
        \textbf{Rol} & Planificador y Programador de la aplicación movil. \\    
        \hline
        \textbf{Contacto} & fduran16@alumnos.utalca.cl \\
        \hline            
        \end{tabular}
    \caption{Descripción Personal FD}
\end{table}


\begin{table}[H]
    \centering
        \begin{tabular}{|l | p{12cm} |}        
        \hline
        \textbf{ID} & IG \\
        \hline
        \textbf{Nombre} & Ignacio Gajardo \\
        \hline
        \textbf{Conocimientos} & Experiencia en lenguaje de programación como Phython, C, C++, C\# , Java, JavaScript, Kotlin y Conocimientos con base de datos MySQL. \\
        \hline
        \textbf{Rol} & Planificador y Programador de la aplicación movil. \\    
        \hline
        \textbf{Contacto} & igajardo16@alumnos.utalca.cl \\
        \hline            
        \end{tabular}
    \caption{Descripción Personal IG}
\end{table}


\begin{table}[H]
    \centering
        \begin{tabular}{|l | p{12cm} |}        
        \hline
        \textbf{ID} & AM \\
        \hline
        \textbf{Nombre} & Alex Molina \\
        \hline
        \textbf{Conocimientos} & Experiencia en lenguaje de programación como Phython, C, C++, C\# , Java y Conocimientos con base de datos MySQL. \\
        \hline
        \textbf{Rol} & Planificador y Programador de la aplicación movil. \\    
        \hline
        \textbf{Contacto} & amolina16@alumnos.utalca.cl \\
        \hline            
        \end{tabular}
    \caption{Descripción Personal AM}
\end{table}



\subsection{Estimación de esfuerzo}
Hemos analizado todos los aspectos posibles que serán parte del desarrollo de nuestro software y que competen a la estimación de esfuerzo, sin embargo, todo lo analizado queda sujeto a modificaciones, debido principalmente a que el proyecto está aún en desarrollo y no poseemos una base o una visión clara del producto final. Tanto a nivel de programación como de diseño a de ser necesaria una frecuente revisión y actualización con cada iteración y avance en este proyecto.

Según lo conversado, pactado y analizado con mis compañeros de trabajo en la primera iteración, los análisis del proyecto se puede apreciar en las siguientes graficas de estimación de puntos de esfuerzo.
\begin{figure}[htbp]
	\includegraphics{1.JPG}
	\includegraphics[scale=0.9]{2.JPG}
\end{figure}

\begin{figure}[htbp]
	\includegraphics{Grafica.png}
	\includegraphics[scale=0.5]{Estimacion de PF.JPG}
\end{figure}

\subsection{Asignación de recursos}

\subsection{Planificación temporal de actividades}