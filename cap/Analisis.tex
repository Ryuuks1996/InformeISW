\section{Análisis}

\subsection{Contexto}
\subsubsection{Descripción General}
Como todo proyecto el paso inicial es un análisis del objetivo a realizar, y este caso no es la excepción. La información utilizada para este análisis consistió en una explicación concisa de los requerimientos necesarios de la aplicación, principalmente mecánicas y características que debe contener la versión final, la cual concluyó que su desarrollo debe tener un énfasis en el funcionamiento básico de esta.\\
El resultado del análisis llevó a una descripción propia del proyecto, el resultado final es una aplicación que funcione como herramienta interactiva basada en una enseñanza de exploración y el contenido que esta acción entrega. La parte lúdica del aprendizaje se ve con la implementación de un sistema inspirado en juegos de mesa donde un numero plural de usuarios toman turnos para realizar acciones y avanzar en un mapa para llegar a una meta final. Cada turno se les entrega a los jugadores una imagen en 360 grados de un área o paisaje en particular y una cantidad de conceptos escogidos al azar (estas palabras son encontradas en una base de datos) con esto los usuarios tendrán que concebir una historia usando los datos mencionados. Al final todos tendrán que votar por otro usuario que consideren haber creado la mejor historia, el usuario ganador avanza un espacio y se sigue la misma idea cada turno hasta que uno llegue al final.
En cuanto a los problemas mencionados en el punto 1.1.2 se tomó la esquematización de estos y se analizaron con profundidad. La base de datos a utilizar debe centrarse en una cantidad reducida de usuarios (cuantos jugadores simultáneos se encuentran), también se llegó a la conclusión de que cada usuario cuenta con su propio dispositivo (un smartphone) por lo que se requerirá uso de internet para facilitar los datos a los usuarios. El análisis del proyecto y de la situación del equipo de desarrollo encuentra que el uso de Android studio (con el lenguaje de kotlin) y la implementación de firebase a este es la opción mas viable.\\
El diseño de la aplicación lleva a la conclusión de comenzar la estructura principal de este (encontrado en el caso de estudio entregado al equipo) dándole mayor importancia la implementación las mecánicas claves recibidas, asegurándose de que estas funcionen sin problemas antes de ampliar o expandir el desarrollo del proyecto.
Lo que respecta a material audiovisual se entrega en las ultimas fases de desarrollo un tiempo 0en particular para la búsqueda e implementación de sonidos o imágenes necesarias, y al igual que los otros casos, dándole mayor importancia a las direcciones entregadas al equipo, en este caso siendo el uso de imágenes en 360 grados en la aplicación.

\subsubsection{Descripción de Clientes y Usuarios:}

\subsection{Especificación de Requerimientos}
\subsubsection{Funciones del Sistema}
Funciones evidentes:
\\	-Registrar nuevo usuario.
\\	-Realizar autentificacion con correo.
\\	-Realizar autentificacion con contraseña.
\\	-Realizar autentificacion con Gmail.
\\	-Mostrar imagenes en 360°.
\\	-Sistema de turnos.
\\	-Generar lista de palabras.
\\	-Multijugador global.\\\\
Funciones ocultas:
\\	-Cargar imagenes en 360°.
\\	-Crear base de datos para palabras.
\\	-Crear base de datos para puntajes.\\\\
Funciones Suérfluas:
\\	-Generar una tabla de puntajes.
\subsubsection{Atributos del Sistema}
-Intuitiva: Debe ser de facil entendimiento para el usuario final, es decir, el ususario debe saber que hacer o como empezar a jugar sin mucha complejidad\\
-Optimizado: La aplicación debe usar recursos del celular solo de ser sumamente necesario\\
-Baja frecuencia de fallos: El Programa no debe presentar mas de cuatro fallos al mes\\
-Facilidad de configuración: Debe ser sencillo para el usuario customizar a su gusto la aplicacion\\
-Facilidad de mantenimiento: A los programadores y gente encargada de mantener la aplicacion dfuncionando se les debe hacer sencillo ingresar al codigo, entenderlo y aplicar las reparaciones\\
\subsubsection{Atributos por Función}

\newpage
\subsection{Actores}
Jugador : La aplicación está hecha para que sea utilizada por usuarios, su último fin es entretener a un público y es quien debiera ser su actor final y principal.\\
2. Escaner QR: Se utiliza para escanear la imagen en 360°.\\
3. FireBase : Permite el contacto entre la aplicación y múltiples herramientas como lo son la base de datos, el guardado de respaldo, etc. Además hace nexo entre la aplicación y los siguientes actores:\\
a. -Base de Datos : La aplicación debe mantener contacto con la base de datos de forma directa, para poder acceder a las palabras, imágenes, usuarios, etc.\\
b. -Gmail : Herramienta principal para identificar al usuario por medio de correo electrónico.
\subsection{Casos de Uso}
\begin{table}[H]
    \begin{center}
        \begin{tabular}{| l | m{12cm} |}        
        	\hline 
        	Identificador & 1\\
        	\hline
        	Caso de Uso & Jugar una partida de Conquista Turística
360°\\
        	\hline
        	Tipo & Primario\\
        	\hline
        	Descripcion & El jugador recibe por pantalla 4 palabras,
las cuales usa para crear una historia. Los
demás jugadores votan por la creatividad
de la historia y deciden que el jugador
puede avanzar al siguiente punto en el
mapa.\\
        	\hline
        \end{tabular}
    \caption{Asignacion del personal a sus distintos cargos}
    \end{center}
\end{table}
\subsubsection{Caso de Uso Esencial}
\subsubsection{Diagrama de Caso de Uso}
\subsubsection{Contrato}
\subsubsection{Modelo Conceptual}
\subsubsection{Diagrama de Secuencia o Colaboración}
\subsubsection{Priorización}

\subsection{Modelo de Dominio}
\subsubsection{Entidades Reconocidas}
\subsubsection{Modelo de Dominio}
\subsubsection{Matriz de Rastreabilidad}