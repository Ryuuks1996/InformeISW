\section{Análisis}

\subsection{Contexto}
\subsubsection{Descripción General}
Como todo proyecto el paso inicial es un análisis del objetivo a realizar, y este caso no es la excepción. La información utilizada para este análisis consistió en una explicación concisa de los requerimientos necesarios de la aplicación, principalmente mecánicas y características que debe contener la versión final, la cual concluyó que su desarrollo debe tener un énfasis en el funcionamiento básico de esta.
El resultado del análisis llevó a una descripción propia del proyecto, el resultado final es una aplicación que funcione como herramienta interactiva basada en una enseñanza de exploración y el contenido que esta acción entrega. La parte lúdica del aprendizaje se ve con la implementación de un sistema inspirado en juegos de mesa donde un numero plural de usuarios toman turnos para realizar acciones y avanzar en un mapa para llegar a una meta final. Cada turno se les entrega a los jugadores una imagen en 360 grados de un área o paisaje en particular y una cantidad de conceptos escogidos al azar (estas palabras son encontradas en una base de datos) con esto los usuarios tendrán que concebir una historia usando los datos mencionados. Al final todos tendrán que votar por otro usuario que consideren haber creado la mejor historia, el usuario ganador avanza un espacio y se sigue la misma idea cada turno hasta que uno llegue al final.
En cuanto a los problemas mencionados en el punto 1.1.2 se tomó la esquematización de estos y se analizaron con profundidad. La base de datos a utilizar debe centrarse en una cantidad reducida de usuarios (cuantos jugadores simultáneos se encuentran), también se llegó a la conclusión de que cada usuario cuenta con su propio dispositivo (un smartphone) por lo que se requerirá uso de internet para facilitar los datos a los usuarios. El análisis del proyecto y de la situación del equipo de desarrollo encuentra que el uso de Android studio (con el lenguaje de kotlin) y la implementación de firebase a este es la opción mas viable. 
El diseño de la aplicación lleva a la conclusión de comenzar la estructura principal de este (encontrado en el caso de estudio entregado al equipo) dándole mayor importancia la implementación las mecánicas claves recibidas, asegurándose de que estas funcionen sin problemas antes de ampliar o expandir el desarrollo del proyecto.
Lo que respecta a material audiovisual se entrega en las ultimas fases de desarrollo un tiempo 0en particular para la búsqueda e implementación de sonidos o imágenes necesarias, y al igual que los otros casos, dándole mayor importancia a las direcciones entregadas al equipo, en este caso siendo el uso de imágenes en 360 grados en la aplicación.

\subsubsection{Descripción de Clientes y Usuarios:}

\subsection{Especificación de Requerimientos}
\subsubsection{Funciones del Sistema}
\subsubsection{Atributos del Sistema}
\subsubsection{Atributos por Función}

\newpage
\subsection{Actores}

\subsection{Casos de Uso}
\subsubsection{Caso de Uso Esencial}
\subsubsection{Diagrama de Caso de Uso}
\subsubsection{Contrato}
\subsubsection{Modelo Conceptual}
\subsubsection{Diagrama de Secuencia o Colaboración}
\subsubsection{Priorización}

\subsection{Modelo de Dominio}
\subsubsection{Entidades Reconocidas}
\subsubsection{Modelo de Dominio}
\subsubsection{Matriz de Rastreabilidad}